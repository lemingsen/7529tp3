\documentclass[../tp3_grupo404.tex]{subfiles}

\graphicspath{{\subfix{../out/}}}

\begin{document}

Una reducción polinomial es un procedimiento por el cual se transforma una instancia de un problema
cuya solución no es conocida en otra instancia de un problema el cual si se sabe resolver.
Luego se transforma la solución obtenida para el problema conocido en una solución para el problema
original.

Si las transformaciones se realizan en tiempo polinomial, entonces es una reducción polinomial.

Notación: Dados los problemas X e Y, se dice $Y \leq_p X$ si Y es polinomialmente reducible a X.
La reducción polinomial puede utilizarse como \textquote{caja negra} para resolver problemas
cuya solución es desconocida. También como medida de complejidad en el caso de saber la
complejidad de alguno de los problemas.
Si $Y \leq_p X$, entonces podemos decir que X es \textquote{al menos tan difícil de resolver} que Y.


% FIN DEL DOCUMENTO (SECCIÓN P3.1)
% NO BORRAR POR ACCIDENTE NI ESCRIBIR COSAS ABAJO
\end{document}