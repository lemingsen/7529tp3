\documentclass[../tp3_grupo404.tex]{subfiles}

\graphicspath{{\subfix{../out/}}}

\begin{document}

Una \textbf{reducción algorítmica} es un procedimiento por el cual se transforma una instancia
de un problema original (o \emph{reducible}), en otra instancia de otro problema (\emph{reducido})
que puede resolverse por un algoritmo dado (a modo de \textquote{caja negra}).
Y luego se transforma la solución obtenida en una solución del problema original.

Eso puede llevarse a cabo porque:
\begin{itemize}
    \item No se sabe resolver el problema original y sí el reducido.
    \item Se conoce un algoritmo para resolver el problema original,
        pero el algoritmo para reducir el problema reducido tiene
        una menor complejidad.
\end{itemize}

Cuando estas transformaciones se realizan en tiempo polinomial,
entonces decimos que es una \textbf{reducción polinomial}.

La reducción polinomial también es de utilidad como \textbf{medida de complejidad} en el caso de
saber la complejidad de alguno de los problemas. Por ejemplo, dados los problemas $X$ e $Y$,
se dice $Y \leq_p X$ si $Y$ es polinomialmente reducible a $X$. Entonces, si $Y \leq_p X$,
entonces podemos decir que $X$ es \underline{al menos} tan difícil de resolver que $Y$.

% FIN DEL DOCUMENTO (SECCIÓN P3.1)
% NO BORRAR POR ACCIDENTE NI ESCRIBIR COSAS ABAJO
\end{document}