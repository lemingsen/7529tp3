\documentclass[../tp3_grupo404.tex]{subfiles}

\graphicspath{{\subfix{../out/}}}

\begin{document}

\noindent\textbf{1. Qué podemos decir de A si utilizamos NA para resolver el problema B
(asumimos que la reducción realizada para adaptar el problema B al problema A es polinomial)}

$B \leq_p A$. Que $A$ es al menos tan difícil de resolver como $B$.
No me dice nada sobre la complejidad de $A$.

\vspace{15mm}\noindent\textbf{2. Qué podemos decir de A si utilizamos NB para resolver el problema A
(asumimos que la reducción realizada para adaptar el problema A al problema B es polinomial)}

$A \leq_p B$. Se puede decir que $A \subseteq P$.
Si $B$ es igual o más difícil de resolver que $A$, y $B$ es $P$; entonces $A$ tiene que ser $P$.

\vspace{15mm}\noindent\textbf{3. ¿Qué pasa con los puntos anteriores si no conocemos la complejidad de B,
pero sabemos que A es NP-C?}

A es NP-C $\implies$ A es NP-H y A es NP.

\begin{description}
    \item[1) $B \leq_p A$] Puedo decir que B es al menos NP.
    \item[2) $A \leq_p B$] Entonces puedo decir que B es NP-Hard.
\end{description}
% FIN DEL DOCUMENTO (SECCIÓN P3.3)
% NO BORRAR POR ACCIDENTE NI ESCRIBIR COSAS ABAJO
\end{document}