\documentclass[../tp3_grupo404.tex]{subfiles}

\graphicspath{{\subfix{../out/}}}

\begin{document}

Un problema es NP-Completo si pertenece a NP y también pertenece NP-Hard,
lo cual quiere decir que es uno de los problemas más difíciles de resolver dentro de NP.
Esta condición de ser de los problemas más difíciles de resolver dentro de NP
los hace importantes ya que si se encontrara la solución a un problema NP-Completo
en tiempo polinómico, se probaría que $P = NP$.

Esto ocurre dado que sabemos que $P \subseteq NP$,
\footnote{ Si se tiene un algoritmo P que resuelve un problema de decisión,
puede usarse como "algoritmo certificador" NP, para verificar la solución dada sea correcta
simplemente comparando que sean la misma. Por esto, todo algoritmo de decisión P es también NP;
pero no necesariamente al revés.}
y si se pudiera probar que el problema más difícil de NP puede solucionarse en tiempo polinomial,
entonces todos los problemas de NP se podrían solucionar en tiempo polinomial
y por lo tanto $NP \subseteq  P$.

Finalmente, si $NP \subseteq P$ y $P \subseteq NP$, entonces $P = NP$.

% FIN DEL DOCUMENTO (SECCIÓN P3.2)
% NO BORRAR POR ACCIDENTE NI ESCRIBIR COSAS ABAJO
\end{document}