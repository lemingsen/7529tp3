\documentclass[../tp3_grupo404.tex]{subfiles}

\graphicspath{{\subfix{../out/}}}

\begin{document}

El problema enunciado consiste en en encontrar el número de equipos de socorro necesarios en las estaciones 
ferroviarias de los distintos ramales y subramales de la red ferroviaria de un país de manera tal que ante 
una emergencia en una estación haya un equipo de socorro en dicha estación, o en el peor de los casos en 
una estación vecina que tenga un trayecto directo entre ellas dos. Además el número de equipos mínimo 
necesario debe ser menor a $k$.

Analizando todo esto, se puede ver que la red ferroviaria formará un grafo no dirigido, en el que cada 
vértice será una estación y las aristas serán los caminos entre estaciones. Este grafo probablemente 
será disconexo, ya que no necesariamente haya comunicación entre todos los ramales.

Por ende, si representamos la red ferroviaria como un grafo no dirigido $G(V,E)$ nuestro problema 
consistirá en encontrar el menor conjunto de vértices (estaciones) $V'$ de tamaño $k'$ de manera tal 
que cualquier vértice de $V$ pertenezca a $V'$ o en su defecto sea adyacente a al menos un vértice que 
pertenezca a $V'$. Además se debe cumplir con que $k'$ sea menor a $k$.

Un set dominante en un grafo no dirigido $G(V,E)$ es un subconjunto de vértices $V'$ tal que cada vértice 
en el grafo pertenece a $V'$ o es adyacente a algún vértice en $V'$. Dado un grafo $G$ y un número entero $k$, 
el problema consiste en encontrar un set dominante de tamaño $k$ para un grafo dado $G$. El número 
dominante $\gamma(G)$ es el cardinal del menor conjunto dominante de $G$.

Por ende, en nuestro problema nos están pidiendo el número dominante $\gamma(G)$ para un problema de set 
dominante y que este sea menor a $k$. 

\begin{itemize}
    \item Dado $G=(V,E)$:
    \item Donde $k$ tamaño máximo del conjunto dominante
    \item $T$ certificado = subconjunto de nodos de $V$
    \item Y puedo verificar en tiempo polinomial:

    Si $(T \leqslant k 	\land (	\forall v \in V, v \in T \lor {v,w} / w  \in T)) \implies \in NP$
\end{itemize}








Si probamos que el problema de set dominante es $NP-Completo$ quedará demostrado entonces que este problema también lo es.


% FIN DEL DOCUMENTO (SECCIÓN Reentrega P2.1)
% NO BORRAR POR ACCIDENTE NI ESCRIBIR COSAS ABAJO
\end{document}