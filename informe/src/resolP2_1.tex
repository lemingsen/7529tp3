\documentclass[../tp3_grupo404.tex]{subfiles}

\graphicspath{{\subfix{../out/}}}

\begin{document}

El problema de situar los equipos de socorro revuelve en torno a el alcance de estos con respecto al conjunto de las estaciones. Se busca que k equipos de emergencia alcancen para que toda estación esté cubierta. Dado que un equipo sólo puede interactuar con su estación o las adyacentes a esta, el problema resulta ser una variante del problema “determinar set dominante”. En este caso, este problema está restringido a determinar un set dominante S con hasta k elementos.

\textquote{Un "set dominante" S de un grafo G es un subconjunto de vértices de dicho grafo,
tales que cada vértice del grafo que no se encuentre en el subconjunto S, es adyacente a aquellos que sí.}
\footnote{Wayne Goddard, Michael A. Henning. «Independent domination in graphs: A survey and recent results». Discrete Mathematics, Volume 313, Issue 7, 2013, pág. 839-854,
ISSN 0012-365X, https://doi.org/10.1016/j.disc.2012.11.031}

El algoritmo solucionador, a grosso modo, debe determinar un set dominante S, de límite de tamaño k,
el cual resulte ser el posicionamiento preferible de los equipos de socorro. El problema de
\textquote{hallar set dominante} fue probado NP-Completo (consecuencia de que Richard M. Karp determinó
que \textquote{set cover problem} es NP-Completo, ocasionada debido a que que toda instancia de
dominating set [“hallar set dominante”] es reducible a una de set cover problem y viceversa).
Por lo tanto, y siendo que el algoritmo es básicamente una solución del problema “hallar set dominante”,
el problema de ubicar los equipos de socorro es NP-Completo.

% FIN DEL DOCUMENTO (SECCIÓN P2.1)
% NO BORRAR POR ACCIDENTE NI ESCRIBIR COSAS ABAJO
\end{document}